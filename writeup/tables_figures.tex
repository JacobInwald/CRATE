
\renewcommand{\t}[1]{\texttt{#1}}

\newcommand{\stackFrameTable}{
    \begin{tikzpicture}[%
        auto,
        block/.style={
          rectangle,
          draw=black,
          thick,
          text width=7.5em,
          align=center,
          minimum height=2em,
        },
        dotsstyle/.style={
            rectangle,
            thick,
            text width=7.5em,
            align=center,
            minimum height=2em
          }
    ]
        \node (dots1) [dotsstyle] 
            {$\qquad$ \dots $\qquad$};
        \node (args2) [block, below=-1pt of dots1] 
            {argv[1]};
        \node (args1) [block, below=-1pt of args2, rectangle]
            {argv[0]};
        \node (sebp) [block, below=-1pt of args1, rectangle]
            {saved \%ebp};
        \node (rbp) [block, below=-1pt of sebp, rectangle]
            {return pointer (\%rbp)};
        \node (l1) [block, below=-1pt of rbp, rectangle]
            {local var 1};
        \node (l2) [block, below=-1pt of l1, rectangle]
            {local var 2};
        \node (dots2) [dotsstyle, below=-1pt of l2] 
            {$\qquad$ \dots $\qquad$};

        \draw [thick, rounded corners=.5em] 
        ([xshift=0.4pt]dots1.north west) rectangle ([xshift=-0.4pt]dots2.south east);

        \node (ebp) [left=2.5em of sebp] {EBP};
        \draw[->, very thick]        (ebp)   -- (sebp);

        \node (esp) [left=2.5em of l2] {ESP};
        \draw[->, very thick]        (esp)   -- (l2);

        \node (highaddrlabel) [above= 2pt of dots1] {
            High Memory Addresses
        };

        \node (lowaddrlabel) [below= 2pt of dots2] {
            Low Memory Addresses
        };


        \draw[->, very thick] 
            ([xshift=1em]dots2.south east) --
            ([xshift=1em]dots1.north east)  node[midway,below, sloped] () {stack growth};
    \end{tikzpicture}
}

\newcommand{\stackFrameTableCanary}{
    \begin{tikzpicture}[%
        auto,
        block/.style={
          rectangle,
          draw=black,
          thick,
          text width=7.5em,
          align=center,
          minimum height=2em,
        },
        block2/.style={
          rectangle,
          draw=black,
          fill=green!20,
          thick,
          text width=7.5em,
          align=center,
          minimum height=2em,
        },
        dotsstyle/.style={
            rectangle,
            thick,
            text width=7.5em,
            align=center,
            minimum height=2em
          }
    ]
        \node (dots1) [dotsstyle] 
            {$\qquad$ \dots $\qquad$};
        \node (args2) [block, below=-1pt of dots1] 
            {argv[1]};
        \node (args1) [block, below=-1pt of args2, rectangle]
            {argv[0]};
        \node (sebp) [block, below=-1pt of args1, rectangle]
            {saved \%ebp};
        \node (rbp) [block, below=-1pt of sebp, rectangle]
            {return pointer (\%rbp)};
        \node (canary) [block2, below=-1pt of rbp, rectangle]
            {canary value};
        \node (l1) [block, below=-1pt of canary, rectangle]
            {local var 1};
        \node (l2) [block, below=-1pt of l1, rectangle]
            {local var 2};
        \node (dots2) [dotsstyle, below=-1pt of l2] 
            {$\qquad$ \dots $\qquad$};

        \draw [thick, rounded corners=.5em] 
        ([xshift=0.4pt]dots1.north west) rectangle ([xshift=-0.4pt]dots2.south east);

        \node (ebp) [left=2.5em of sebp] {EBP};
        \draw[->, very thick]        (ebp)   -- (sebp);

        \node (esp) [left=2.5em of l2] {ESP};
        \draw[->, very thick]        (esp)   -- (l2);

        \node (highaddrlabel) [above= 2pt of dots1] {
            High Memory Addresses
        };

        \node (lowaddrlabel) [below= 2pt of dots2] {
            Low Memory Addresses
        };

        \draw[->, very thick] 
            ([xshift=1em]dots2.south east) --
            ([xshift=1em]dots1.north east)  node[midway,below, sloped] () {stack growth};
    \end{tikzpicture}
}


\newcommand{\canaryAssembly}{
\begin{figure}[H]
    % \begin{subfigure}[l]{.5\textwidth}
        % \canaryAssembly
\begin{lstlisting}[language=C]
-- unchanged
push   \%rbp             ; preserves previous return pointer
mov    \%rsp,\%rbp       ; save return pointer to %rbp
sub    \$0x40,\%rsp      ; makes sure there's space for local variables
mov    \%edi,-0x34(\%rbp)
mov    \%rsi,-0x40(\%rbp)
-- inserted code
mov    \%fs:0x28,\%rax  ; load canary value into %rax
mov    \%rax,-0x8(\%rbp);
xor    \%eax,\%eax
-- code block for main, left unchanged 
-- inserted code --
mov    -0x8(\%rbp),\%rdx
sub    \%fs:0x28,\%rdx
je     0x4011c9 <main+147>
call   0x401030 <\_\_stack\_chk\_fail@plt>     
-- unchanged 
leave
ret    
\end{lstlisting}

\caption{A normal stack frame without a canary}
        % \end{subfigure}%
        % \begin{subfigure}[r]{.5\textwidth}
        %     \caption{A stack frame with a canary inserted}
        %     \centering
        %     \canaryAssembly
        % \end{subfigure}
        \label{fig:canarycode}
    \end{figure}
}


% \newcommand{\temp}{
% \begin{listing}[h]
%     \centering
%     \caption{x86 Code}
%     \begin{minted}[linenos,frame=single]{nasm}
% jmp    0x401170 <main+74>
% mov    -0x10(%rbp),%rax
% movzbl (%rax),%edx
% mov    -0x8(%rbp),%rax
% mov    %dl,(%rax)
% mov    -0x8(%rbp),%rax
% add    $0x1,%rax
% mov    %rax,-0x8(%rbp)
% mov    -0x10(%rbp),%rax
% add    $0x1,%rax
% mov    %rax,-0x10(%rbp)
% mov    -0x10(%rbp),%rax
% movzbl (%rax),%eax
% test   %al,%al
% jne    0x40114b <main+37>
% lea    -0x30(%rbp),%rax
% mov    %rax,%rsi
% mov    $0x402010,%edi
% mov    $0x0,%eax
% call   0x401030 <printf@plt>
% mov    $0x0,%eax
% leave
% ret
     
%     \end{minted}
% \end{listing}



% \begin{listing}[h]
%     \centering
%     \caption{C Code}
%     \begin{minted}[linenos,frame=single]{c}
% \# include "stdio.h"

% int main(int argc, char *argv[])
% {
%     char buffer[20];
%     char *to = buffer;
%     char *from = argv[1];
%     while (*from)
%     {
%         *to = *from;
%         *to++;
%         *from++;
%     }
%     printf("Your buffer is %s\n", buffer);
% }
%     \end{minted}
% \end{listing}

% }